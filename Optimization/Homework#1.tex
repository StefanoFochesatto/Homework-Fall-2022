
%%%%%%%%%%%%%%%%%%%%%%%%%%%%%%%%%%%%%%%%%%%%%%%%%%%%%%%%%%%%%%%%%%%%%%%%%%%%%%%%%%%%%%%
%%%%%%%%%%%%%%%%%%%%%%%%%%%%%%%%%%%%%%%%%%%%%%%%%%%%%%%%%%%%%%%%%%%%%%%%%%%%%%%%%%%%%%%
% 
% This top part of the document is called the 'preamble'.  Modify it with caution!
%
% The real document starts below where it says 'The main document starts here'.

\documentclass[12pt]{article}

\usepackage{amssymb,amsmath,amsthm}
\usepackage[top=1in, bottom=1in, left=1.25in, right=1.25in]{geometry}
\usepackage{fancyhdr}
\usepackage{enumerate}
\usepackage{listings}
\usepackage{graphicx}
\usepackage{float}
\usepackage{multicol}
% Comment the following line to use TeX's default font of Computer Modern.
\usepackage{times,txfonts}
\usepackage{mwe}
\usepackage{caption}
\usepackage{subcaption}




\makeatletter
\renewcommand*\env@matrix[1][*\c@MaxMatrixCols c]{%
  \hskip -\arraycolsep
  \let\@ifnextchar\new@ifnextchar
  \array{#1}}
\makeatother

\newtheoremstyle{homework}% name of the style to be used
  {18pt}% measure of space to leave above the theorem. E.g.: 3pt
  {12pt}% measure of space to leave below the theorem. E.g.: 3pt
  {}% name of font to use in the body of the theorem
  {}% measure of space to indent
  {\bfseries}% name of head font
  {:}% punctuation between head and body
  {2ex}% space after theorem head; " " = normal interword space
  {}% Manually specify head
\theoremstyle{homework} 

% Set up an Exercise environment and a Solution label.
\newtheorem*{exercisecore}{Exercise \@currentlabel}
\newenvironment{exercise}[1]
{\def\@currentlabel{#1}\exercisecore}
{\endexercisecore}

\newcommand{\localhead}[1]{\par\smallskip\noindent\textbf{#1}\nobreak\\}%
\newcommand\solution{\localhead{Solution:}}

%%%%%%%%%%%%%%%%%%%%%%%%%%%%%%%%%%%%%%%%%%%%%%%%%%%%%%%%%%%%%%%%%%%%%%%%
%
% Stuff for getting the name/document date/title across the header
\makeatletter
\RequirePackage{fancyhdr}
\pagestyle{fancy}
\fancyfoot[C]{\ifnum \value{page} > 1\relax\thepage\fi}
\fancyhead[L]{\ifx\@doclabel\@empty\else\@doclabel\fi}
\fancyhead[C]{\ifx\@docdate\@empty\else\@docdate\fi}
\fancyhead[R]{\ifx\@docauthor\@empty\else\@docauthor\fi}
\headheight 15pt

\def\doclabel#1{\gdef\@doclabel{#1}}
\doclabel{Use {\tt\textbackslash doclabel\{MY LABEL\}}.}
\def\docdate#1{\gdef\@docdate{#1}}
\docdate{Use {\tt\textbackslash docdate\{MY DATE\}}.}
\def\docauthor#1{\gdef\@docauthor{#1}}
\docauthor{Use {\tt\textbackslash docauthor\{MY NAME\}}.}
\makeatother

% Shortcuts for blackboard bold number sets (reals, integers, etc.)
\newcommand{\Reals}{\ensuremath{\mathbb R}}
\newcommand{\Nats}{\ensuremath{\mathbb N}}
\newcommand{\Ints}{\ensuremath{\mathbb Z}}
\newcommand{\Rats}{\ensuremath{\mathbb Q}}
\newcommand{\Cplx}{\ensuremath{\mathbb C}}
%% Some equivalents that some people may prefer.
\let\RR\Reals
\let\NN\Nats
\let\II\Ints
\let\CC\Cplx

%%%%%%%%%%%%%%%%%%%%%%%%%%%%%%%%%%%%%%%%%%%%%%%%%%%%%%%%%%%%%%%%%%%%%%%%%%%%%%%%%%%%%%%
%%%%%%%%%%%%%%%%%%%%%%%%%%%%%%%%%%%%%%%%%%%%%%%%%%%%%%%%%%%%%%%%%%%%%%%%%%%%%%%%%%%%%%%
% 
% The main document start here.

% The following commands set up the material that appears in the header.
\doclabel{Math 661: Homework 1}
\docauthor{Stefano Fochesatto}
\docdate{\today}


\begin{document}


\begin{exercise}{2.1} Consider the feasible region defined by the constrains, 
  \begin{equation*}
    1 - x_1^2 - x_2^2 \geq 0, \sqrt{2} - x_1 - x_2 \geq 0,\text{ and } x_2 \geq 0. 
  \end{equation*}
  For each of the following points, determine whether the point is feasible or infeasible and (if it is feasible)
  whether it is interior to or on the boundary of each of the constraints:
  \begin{enumerate}
    \item $x_a = (\frac{1}{2}, \frac{1}{2})^{T}$.
    \solution Checking constraints, 
    \begin{equation*}
      1 - \left(\frac{1}{2}\right)^2 - \left(\frac{1}{2}\right)^2 = 1 - \frac{2}{4} = \frac{1}{2} > 0
    \end{equation*}
    \begin{equation*}
      \sqrt{2} - (\frac{1}{2}) - (\frac{1}{2}) = \sqrt{2} - 1 > 0
    \end{equation*}
    and clearly $\frac{1}{2} > 0$. Thus $x_a$ is a feasible point, and since we have strict inequalities
    on all the constraints we know that $x_a$ is interior to all constraints (Defined in Chapter 2 p.44). 
    
    \item $x_b = (1,0)^T$
    \solution Checking all constraints, 
    \begin{equation*}
      1 - 1^2 - 0^2 = 1 - 1 =  0
    \end{equation*}
    \begin{equation*}
      \sqrt{2} - (1) - (0) = \sqrt{2} - 1 > 0
    \end{equation*}
    and since $x_2 = 0$ we know that $x_b$ is a feasible point. Note that $x_b$ is on the boundary of 
    $1 - x_1^2 - x_2^2 \geq 0$ and $x_2 \geq 0$, and interior to $\sqrt{2} - x_1 - x_2 \geq 0$.


    \item $x_c = (-1, 0)^T$
    \solution Checking all constraints, 
    \begin{equation*}
      1 - (-1)^2 - 0^2 = 1 - 1 =  0
    \end{equation*}
    \begin{equation*}
      \sqrt{2} - (-1) - (0) = \sqrt{2} + 1 > 0
    \end{equation*}
    and since $x_2 = 0$ we know that $x_b$ is a feasible point. Note that $x_c$ is on the boundary of 
    $1 - x_1^2 - x_2^2 \geq 0$ and $x_2 \geq 0$, and interior to $\sqrt{2} - x_1 - x_2 \geq 0$.

    \item $x_d = (-\frac{1}{2}, 0)^T$ 

  \end{enumerate}


\end{exercise}
\vspace{1in}







\end{document}


































